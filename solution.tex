\documentclass[11pt,a4paper,leqno]{article}
\usepackage{a4wide}
\usepackage[T1]{fontenc}
\usepackage[utf8]{inputenc}
\usepackage{float, afterpage, rotating, graphicx}
\usepackage{longtable, booktabs, tabularx}
\usepackage{verbatim}
\usepackage{eurosym, calc, chngcntr}
\usepackage{amsmath, amssymb, amsfonts, amsthm, bm, delarray}
\usepackage{caption}
\usepackage{tkz-graph}
\usetikzlibrary{arrows,positioning,snakes,shapes,shapes.multipart,patterns,mindmap,shadows}

% \usepackage[backend=biber, natbib=true, bibencoding=inputenc, bibstyle=authoryear-ibid, citestyle=authoryear-comp, maxnames=10]{biblatex}
% \bibliography{bib/hmg}

\usepackage[unicode=true]{hyperref}
\hypersetup{colorlinks=true, linkcolor=black, anchorcolor=black, citecolor=black, filecolor=black, menucolor=black, runcolor=black, urlcolor=black}
\setlength{\parskip}{.5ex}
\setlength{\parindent}{0ex}

\theoremstyle{definition}
\newtheorem{exercise}{Exercise}
\renewcommand{\theenumi}{\roman{enumi}}

% Set this counter to "first exercise of the week minus one".
\setcounter{exercise}{0}

\begin{document}

\begin{center}
    \begin{large}
        \textbf{
        Effective programming practices for economists\\
        Universität Bonn, Winter 2019/20 \\[2ex]
        Exercise solution\\[2ex]
        Your name here
        }
    \end{large}
\end{center}


\begin{exercise}
    ~ % do not delete this innocent tilde unless you put text here
    \begin{enumerate}
        \item Your answer here\ldots
        \item {\ldots} and here.
    \end{enumerate}
\end{exercise}


% \printbibliography

\end{document}
